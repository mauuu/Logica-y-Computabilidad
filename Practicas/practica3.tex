\begin{enumerate}

\item 
\begin{enumerate}
\item No se que es, creo que no es nada porque por definici\'on $f_2$ es binaria pero aqui se la usa como si fuese una funci\'on unaria. Adem\'as el cuantificador existencial est\'a cuantificando una f\'ormula cuando deber\'ia cuantificar una variable
\item Es una f\'ormula ya que $f_1$ es unaria y $f_2$ es binaria
\item No es una f\'ormula debido a que se est� cuantificando una constante $C$ con el cuantificador existencial, cuando solo se deben cuantificar variables �Es un t\'ermino?
\item Idem anterior
\item El $\exists x$ aparece dos veces en la misma sentencia, esta mal formado, no es f\'ormula �Es un t\'ermino?
\item Es f�rmula
\end{enumerate}
\medskip
%%Segundo Ejercicio
\item
\begin{enumerate}
\item En la primera parte x aparece ligada en el $\forall x$, tambi�n $y$ aparece ligada pues ambas estan siendo afectadas por cuantificadores
\item En $(\exists xP(y,y))\ y$ aparece libre, $x$ NO debido a que est� siendo afectada por un cuantificador, luego en $\exists yP(y,z)\ y$ aparece ligada y $z$ libre
\item Veamos que $x$ aparece ligada ya que est� afectada por el cuantificador existencial, de modo que en subsiguientes apariciones $x$ ya estar� ligada. Analizando la primera parte de la conjunci�n $(\exists yP(x,x))\ y$ aparece ligada pero en la segunda parte $(P(x,y))\ y$ aparece libre pues el cuantificador solo afecta a la primera parte de la conjunci�n
\item Bastante parecido al anterior
\end{enumerate}\smallskip

\medskip
%%Tercer Ejercicio
\item
\begin{enumerate}
\item No es una interpretaci�n v�lida ya que $f_1$ va de los $ \mathbb{N} \rightarrow \mathbb{R}$ mientras que el universo de interpretaci�n $U_I=\mathbb{N} $, absurdo ya que la imag�n de la funci�n corresponde a algo que cae fuera del universo de interpretaci�n
\item Es v�lida
\item Si el universo de interpretaci�n $U_I$ son los \Naturales entonces esta no es una interpretaci�n v�lida debido a que las constantes $c_I = d_I = 0$ y el 0 esta fuera del universo de interpretaci�n. Si se esta incluyendo al 0 dentro de los \Naturales, es decir \NaturalesYCero entonces la interpretaci�n es v�lida. Adem�s la imagen de la funci�n $g_1(n,n) = n^2 -n$ es siempre $\geq 0$ \paraTodo $n \in \mathbb{N} \cup 0\ $y puede demostrarse por inducci�n
\end{enumerate}\smallskip
%%Cuarto Ejercicio
\item
\begin{enumerate}
\item Para todo $x,y\ $ si $x\leq y\ $ existe un numero $z$ tal que $x\leq z\ $ y $z\leq y.\ $ Me esta diciendo en los reales dados dos numeros siempre voy a poder encontrar un numero mayor al primero y menor al segundo.
\item Todos los d�as nace un esclavo
\item Para todos los n�meros, si son pares, vale que su suma da como resultado un n�mero impar
\end{enumerate}\smallskip
%%Quinto Ejercicio
\item
\begin{enumerate}
\item Hay una persona que quiere a todas
\item Toda persona tiene alguien que lo quiera
\item Para toda persona que quiera a alguien, hay una persona que lo quiere a �l.
\item Hay alguien que no quiere a nadie
\end{enumerate}\smallskip
%%Sexto Ejercicio
\item
\begin{enumerate}
\item
\end{enumerate}
\end{enumerate}